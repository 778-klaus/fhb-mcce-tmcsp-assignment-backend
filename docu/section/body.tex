\section{Browser IDE Vergleich}
In diesem Abschnitt werden die drei Browser IDEs
github.dev \cite{githubDevWebsite},
stackblitz.com \cite{stackblitzcomWebsite},
und
gitpod.io \cite{gitpodioWebsite}
grob miteinander vergleichen.
Dabei wird auf den Editor und die wichtigsten Features eingegangen.

\subsection{github.dev}
% TODO

\subsection{stackblitz.com}
% TODO

\subsection{gitpod.io}
% TODO


\clearpage
\section{Dokumentation der Projektumsetzung}

\subsection{Schritt-für-Schritt-Anleitung}
Zuerst wird das Basis-Repository 
\url{https://github.com/t-stefan/FHB-Assignment-Backend}
im GitHub Webinterface
\url{https://github.com/new/import} 
importiert.

% TODO: search & replace all  thomasstxyz/fhb-mcce-tmcsp-assignment-backend  with <account/repository>

\noindent
Als nächstes müssen die GitHub Actions für das Repository unter
\url{https://github.com/thomasstxyz/fhb-mcce-tmcsp-assignment-backend/settings/actions}
aktiviert werden.
Wir wählen hierfür den Punkt \verb|Allow all actions and reusable workflows| aus
und drücken anschließend auf \verb|Save|. \\

\noindent
Nun kann zu \url{https://github.com/thomasstxyz/fhb-mcce-tmcsp-assignment-backend/actions/new}
navigiert werden.
Hier wählen wir den Workflow \verb|Node.js by GitHub Actions| 
(\url{https://github.com/thomasstxyz/fhb-mcce-tmcsp-assignment-backend/new/main?filename=.github%2Fworkflows%2Fnode.js.yml&workflow_template=node.js}).
Nun öffnet sich ein beispielhafter Workflow als neue Datei im Editor.
Mit \verb|Start commit| speichern wir diesen.
Anschließend kann zu \url{https://github.com/thomasstxyz/fhb-mcce-tmcsp-assignment-backend/actions/workflows/node.js.yml}
navigiert werden und der erste Durchlauf unseres Workflows live angesehen werden.



